\documentclass[openany]{article}

\usepackage[utf8]{inputenc}
\usepackage{dirtytalk}
\usepackage{amsmath}
\usepackage{mathtools}  
\mathtoolsset{showonlyrefs} 
\usepackage{graphicx}
\usepackage{mdframed}
\usepackage{lipsum}
\usepackage{cancel}
\usepackage{systeme}
\usepackage{pgfplots}
\usepackage{enumitem}
\usepackage{textcomp}
\usepackage{geometry}
\usepackage{siunitx}
\usepackage{gensymb}  % Symbol for degrees
\usepackage{wrapfig}  %floating options using wrapfigure enviroment
\usetikzlibrary{arrows}
\geometry{a4paper}
\graphicspath{ {./res/} }
\usepackage{float}
\restylefloat{table}
\newcommand{\comment}[1]{%
	\text{\phantom{(#1)}} \tag{#1}
}
 \title{\line(1,0){450}\\An AI-assisted decision making system for thyroid nodule classification \\ Meetings Logbook \\ \footnotesize University of Reading  \\\line(1,0){450}}
\usepackage{pgfplots}
\author{Stefanos Stefanou}

\begin{document}
	\maketitle
	\section{Brief Table}
		\begin{table}[H]
			\begin{tabular}{lllll}
				29/09/2020 & Initial comments on PID &  &  &  \\
				6/10/2020  & Final Comments on PID (title)&  &  &  \\
				13/10/2020 & Literature report and setting goals&  &  &  \\
				20/10/2020 & Comments on the literature report/Are the DDIT a suitable dataset?&  &  &  \\
				27/10/2020 & First results, DDIT problems and solutions, next target: TDID and formal classification process&  &  &  \\
				03/11/2020 & Week 6, No meeting was held&  &  &  \\
				10/11/2020 & Multiple objectives achieved, promising results, objective towards service/UI perspective set&  &  &  \\
				17/11/2020 & Not Perfomed Due to personal reasons&  &  &  \\
				24/11/2020 & Multiple objectives achieved, promising results, objective towards service/UI perspective set&  &  &  \\				
				1/12/2020  & Not Performed/No updates, Develepoment of the platform continues on schedule&  &  &  \\				
				1/12/2020  & No Meeting was held, because of no further updates. System develepoment continues on planwith a demo scheduled for the next meeting&  &  &  \\				
				8/12/2020  & Minor delays in delivery of Frontend, developement resumes as normal, xmas objectives agreed, Top priority the presentation and the Frontend&  &  &  \\				
				11/01/2021 & Frontend is ready, and the majority of the study for start developing backend is done, some rescheduling done, some tasks will be pushed back	&  &  &  \\				
				25/01/2021 & Part of the backend is ready, develepoment continues on plan	&  &  &  \\				
			
		\end{tabular}
		\end{table}
	\section{Analytic Report}
		\subsection{29/09/2020}
			Some advice on good report writing is given by the supervisor, Mrs. Liang. Additionally, some typos regarding the PID draft and some clarification on the various subsections of PID Template are also given. Some questions on context and expectations are also answered.
		\subsection{6/10/2020}
			Some final comments on PID regarding the title and the type of the system (web or standalone) are given. The PID is now at the 'Ready' state and gets approval from Mrs. Liang.
		\subsection{13/10/2020}
			Some information regarding the literature report is discussed, and short term goals are set. The first goal is to perform some minor transformations into the DDIT dataset and building the first classifier in python and various technologies. The study-first-write-later approach to this project is agreed. Under this approach, a period of non-rapid progress will exist. The aforementioned approach will allow the student to properly study the underlying subject and its mathematical formalism before he engages with the topic.
		\subsection{20/10/2020}
			Some guidance regarding the literature report and how it needs to be more 'research-based' was given. Additionally, we discussed the importance of proving that DDIT is a suitable database for our needs. Plan on how we will prove it based on a simple python-classifier. This goal is to be met by our next meeting.
		\subsection{27/10/2020}
			The first results regarding the viability of the DDIT dataset are discussed. The integration library is tested, and the first classifier using scikit-learn is implemented. 97/98 of samples can be successfully predicted using this approach. Some significant problems are also pointed out, as the inconsistency of the dataset in certain areas and the very few valid points that can be used. The fact that maybe more datasets are needed is pointed out. The next target is defined. Next week, the TDID dataset needs to be tested, and the formal classification process needs to be followed, splitting our data points in an 80-20\% setup.
		\subsection{03/11/2020}
			Week 6, No meeting was held.
		\subsection{10/11/2020}
			Opening, the results of the two previous weeks are being presented.
			\begin{itemize}
				\item Integrate TDID into the codebase
				\item 80-20 setup : 75-80 \%accuracy
				\item Study : Complete Revision of : Probability and Statistics
				\item Progress on the topics study
				\item 5 papers found, studied, and added in literature report
				\item TIRADS idea, multiple labels classification. Is it viable?
			\end{itemize}	
			After the initial presentation, some feedback in the proccess is given, and the next objectives are set, namely
			\begin{itemize}
				\item Try different classifiers ,variance of with parameters, compare results
				\item Multiple label classification. Compare results
				\item Introduction of complex image filters. Compare results
				\item Paper prototype, what are the features that we want to support?
				\item Define a draft of Use cases, Interactions
				\item Service perspective of the project, maybe a mock of UI
				\item What are the end users?
			\end{itemize}
		\subsection{17/11/2020}
			No meeting was held due to personal reasons
		\subsection{24/11/2020}
			What objectives have been completed?
			\begin{itemize}
				\item Try different classifiers ,variance of with parameters, compare results(NO)
				\item Multiple label classification. Compare results(NO)
				\item Introduction of complex image filters. Compare results(NO)
				\item Paper prototype, what are the features that we want to support?(YES)
				\item Define a draft of Use cases, Interactions(YES)
				\item Service perspective of the project, maybe a mock of UI(YES)
				\item What are the end users?(YES)
			\end{itemize}
			Some notes from the supervisor. 
			\begin{itemize}
				\item Paper prototype:align with the universitys report template
				\item Define a draft of Use cases, Interactions :Write for those in the report
				\item Service perspective of the project: Architecture okay, create a mini presentation for the doctors, we need feedback
				\item What are the end users: only hospitals, agreed :) 
			\end{itemize}
			Next time objectives
			\begin{itemize}
				\item Align with the universitys report termplate
				\item mini presentation
				\item write about the user interactions
				\item write about the softwares architecture in the report
				\item minimal functional version of the software
				\item (carryover) Try different classifiers ,variance of with parameters, compare results
				\item (carryover)Multiple label classification. Compare results
				\item (carryover)Introduction of complex image filters. Compare results
			\end{itemize}
		\subsection{1/12/2020}
			No Meeting was held, because of no further updates. System develepoment continues on plan
			with a demo scheduled for the next meeting
		\subsection{8/12/2020}
			Objectives
			\begin{itemize}
				\item (1)Align with the universitys report termplate(DONE)
				\item (2)mini presentation(To be completed)
				\item (3)write about the user interactions (DONE)
				\item (4)write about the softwares architecture in the report(DONE)
				\item (5)minimal functional version of the software(In progress)
				\item (6)(carryover) Try different classifiers ,variance of with parameters, compare results (Postponed)
				\item (7)(carryover)Multiple label classification. Compare results(Postponed)
				\item (8)(carryover)Introduction of complex image filters. Compare results(Postponed)
			\end{itemize}
			Because of university deadlines, the develepoment of a minimal functional version of the software is postponed untill 22th of december.
			Presentation to be completed by the next meeting, Tasks (1),(3),(4) are completed and tasks (6),(7),(8) are postponed untill next year.
			Top priority, the delivery of the frontend for the next few weeks.
			
		\subsection{11/01/2021}
			Things Achieved on christmas
			
			\begin{itemize}
				\item minimal functional version of frontend
			\end{itemize}
			Frontend is ready, and the majority of the study for start developing backend is done. for the next few weeks the first microservice(backend-1) will be developed. 
			With the new scheduling, tasks(6),(7) and (8) will be postponed untill after the end of the develepoment of backend-2.
		\subsection{25/01/2021}
			Part of the backend is ready, develepoment continues on plan,20 Tasks are complete, demo of the following
			\begin{itemize}
				\item Login System
				\item Signup System
				\item Profile Endpoint
				\item Django endpoint
				\item Heroku deployment
				\item Split into multiple projects for ease of use and proffesionalism
			\end{itemize}
			The following plan was aggreed
			\begin{itemize}
				\item 20-30 Jan: 				Backend Microservice 1 (Information) + Migration FE/BE
				\item 1-10 Feb:  				Backend Microservice 2 (Task)+Migration
				\item 10 Feb-15 March: 		    Study of machine learning patterns + University Deadlines
				\item 15 March - 15 April:      Complex models,image filters+prediction focused period
				\item 15 April-20April:         Presentation work and safety padding
			\end{itemize}
		\subsection{01-02}
			\begin{itemize}
				\item All the required functionality of Information Backend is ready(demostration performed),
				\item Part of FE-BE Migration is ready. Develepoment continues on plan
				\item Profile Section Implemented
				\item Notifications System Implemented
				\item Login System Implemented
				\item Nessesary study materials completed
				\item Support NINO on patients
			\end{itemize}
			Targets for next week
			\begin{itemize}
				\item Information Backend Ready
				\item Migration Complete
				\item The basic model to be runnable, and shuffling mechanism to be implemented and tested. 
				\item PostDoc meeting to be held next week with advices on improving the model
				\item Doctors meeting to be scheduled
			\end{itemize}
		\subsection{08-02}
			\begin{itemize}
				
				\item Information Backend is ready on schedule
				\item Migration FE-BE is ready on schedule(Demostration performed)
				\item The basic model to be runnable, and shuffling mechanism implemented
				\item PostDoc meeting held. Introduction to the system, we disqussed various approaches on enchacing the system
				\item Doctors meeting for feedback to be done in 2 weeks time
			\end{itemize}
			Some notable tasks following
			\begin{itemize}
				\item Comments on patient (comments on scan will follow later)
				\item Search by nino
				\item About page
				\item Title+logo
				\item Error handling
				\item Refactoring
				\item Patient add
				\item Patient list
				\item Notifications pill
				\item Profile username upper left
				\item Contract testing
				\item Shufflng mechanism on the basic model
				\item + 2 bugs fixed
			\end{itemize}
			For Next week
			\begin{itemize}
				\item Task-Backend to be implemented
				\item Task-Backend - FE Migration to be performed
				\item Finalization of user interactions
				\item Start working on the Doctors presentation
				\item Start working on the Project poster presentation for February 26th
			\end{itemize}
		\subsection{16-02}
			25 Completed Tasks
			\begin{itemize}
				\item Full Cycle is implemented
				\item All interactions are implemented
				\item Comments on Patient and on Scan
				\item Patient View and Scan view
				\item View All scans for a given patient
				\item Form alignment and Table alignment, all is uniform 
				\item Notification system completed
				\item Change ST marys hospityal to RBH
			\end{itemize}
			Whats Left
			\begin{itemize}
				\item Image upload and storange on TE (mocked)
				\item Classification Handling on TE (will use thespian actors for those)
				\item Putting an actual model
				\item Studing my last 2 books
				\item Complex models migration
				\item Update the final report
			\end{itemize}
			
			How? 
			from now untill 15 march, will go to low-productivity mode
			1) Will finish poster video
			2) Will finish presentation
			3) Will Implement 1,2,3,6,4
			
			Those to be expected by 15 march, by implementing those i will have 4 weeks for 
			complex model and finish the project(quite good programme)
			
			What low-productivity means?
			instead of 5*7 = 35hrs/week
			2*7 = 14hrs/week
			
			Some Suggestions have been given by Mrs Liang
			\begin{itemize}
				\item Implement Hint's for making the system user friendly
				\item Better Logo and Naming
				\item Agree/Disagree functionality
			\end{itemize}
			Those Has been agreed to have been implemented by the next week.
			Additionally, the next week, a doctors meeting will be held and i will present my work for receiving useful feedback
\end{document}

