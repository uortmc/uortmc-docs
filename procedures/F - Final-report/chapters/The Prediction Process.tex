\chapter{The Prediction Proccess}
\label{prediction-process}
	\section{Introduction}
		In this section we are going to disquss the algorithms involved in the prediction proccess, a process running inside the Prediction
		Service, as well as the libraries and the schematics of the predictors themselfs.\par
		\subsection{Technology Stack}
			The prediction proccess uses several libraries to achieve its goals; the most important ones are listed below.
			\begin{itemize}
				\item Numpy linear algebra library.
				\item Scikit-learn machine learning library.
				\item torch and torchvision deep learning library.
			\end{itemize}
			\subsubsection{Numpy}
				Numpy is a python compatible library, adding support for matrices and multi-dimentional arrays, along with an 
				extesive set of routines, for linear algebra operations. We use Numpy for various image transformations, spatial domain
				image filters and dataset handling
				
			\subsubsection{Scikit-learn}
				Scikit-learn is a open-source and free machine learning library compatible with python. Includes a rich ecosystem of
				regression and clustering algorithms, including but not limited to, random forests,vector machines, gradient boosting, 
				k-means and others. It is designed to interoperate with numerous Python numerical and scientific libraries such as NumPy 
				and SciPy.
			\subsection{pytorch and pytorchvision}
				pytorch and pytorchvision are the python bindings of the famous torch library. Torch is anm free and open source deep 
				learning and natural language processing library, developed by Facebook's AI Research Lab.
		\subsection{Algorithms}
			
		
		
		
		Our Application currently supports 2 algorithms for thuroid nodule classification, named
		\begin{itemize}
			\item C-Support Vector Machine, codenamed SVC v1.
			\item Residual Deep Neural Network, codenamed RESNet v18
		\end{itemize}
	

