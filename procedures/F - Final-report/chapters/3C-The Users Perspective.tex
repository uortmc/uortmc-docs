\chapter{Users Perspective}
\label{users_perspective}
	\section{Introduction}
		In this section, we will start our exploration of the application and its infrastructure and features. As the nature of the requirements of this
		the system is complicated. Unavoidably the system will be complex as well. Taking this into a consideration, we will follow a natural top-to-bottom
		approach explaining its internals, starting as our end-users and seeing the system as a black box. In this section, we will analyze its 
		functionality from the user's perspective. Our definitions will be rigorous as we will use UML Case diagrams for that purpose.
		on 
	\section{Our Users}
		In Chapter  we briefly mentioned our system's purpose as ...
		\begin{mydef}
			An AI-assisted decision making system for thyroid nodule classification.
		\end{mydef}
		As we are going to see later, various scientific methods are applied, taking into account 
		multiple parameters, in order to produce a probabilistic result. This fact implies that the
		end-user will be an expert on Radiology, to understand the terms, and carefully interpret the 
		results. From now on we assume that our system users will be experianced Radiologists
	\section{Our Users}
		In Chapter we briefly mentioned our system's purpose as ...
		\begin{mydef}
			An AI-assisted decision making system for thyroid nodule classification.
		\end{mydef}
		As we are going to see later, various scientific methods are applied, taking into account 
		multiple parameters, in order to produce a probabilistic result. This fact implies that the
		end-user will be an expert on Radiology, to understand the terms, and carefully interpret the 
		results. From now on we assume that our system users will be experianced Radiologists
		
	



