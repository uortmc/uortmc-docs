\chapter{Discussion, Conclusion and Future work}
\label{ch:lit_rev}
	\section{Introduction}
		It is for certain that this first iteration is not perfect by any means, much work need to be done for this project to become 
		a contributing tool for the academic community, but the results are iitial results are promising. After working closely with
		my supervisor, Mrs Liang, We started disqussing the possibility of continuing the develepoment of this project if sufficient
		funding is found, if this is the case then this tool has the capability to speed up the Thyroid nodule classification research even
		more.
	\section{Future work}
		There are multiple areas that we can focus on to improve the application and its performance, some proposed ideas are presented below
		\subsection{Improving the SVC Prediction Model}
			There are multiple chances for improvement when comes to the prediction model, for improving its accuracy results. A proposed
			area of improvement is to, instead of binary classifing the given ultrasound image, to also find the exact thuroid nodule borders
			within the image, by taking advantage the additional metadata given by the TDID dataset.
		\subsection{Multiple prediction models and vote system}
			A different idea that was proposed, is the utilisation of multiple prediction models, and the aggregation into a single view, increasing
			the accuracy results, by the prediction answer of every algorithm, and based on an weighted system of voting, to determine the actual
			prediction.
		\subsection{Application monitoring using Graphana}
			A proposal coming from the fact that the application does not have any monitoring capability at the moment. By adding monitoring
			we can have real-time metrics about the applications performance and status, This will be a great tool for the administrator operator as it
			will provide the capability to react in sudden surges in demand, or possible errors quicker.
		
		
...


